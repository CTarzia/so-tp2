\section{Conclusiones}

%Se discuten los resultados
%observados en la sección anterior y se da un cierre general.

% Distribuidos 

% Elegir una dificultad adecuada difici...

% Cambiar la dificultad segun cantidad de nodos (similar a Bitcoin)

% 

Concluimos que elegir una dificultad
adecuada es complejo, ya que, por un lado 
no queremos que los bloques tarden mucho en agregarse 
a la cadena, y por el otro, no queremos que sea demasiado baja, para que
no haya muchos conflictos. La conflictividad depende a su 
vez de la cantidad de nodos en la red, lo cual puede no 
conocerse a priori, o puede ser variable.

Es por estas razones que sería interesante aplicar
una política que modifique la dificultad en base a
estos parámetros. Existen criptomonedas que realizan
políticas de este estilo para regular la creación de bloques.


Este protocolo, como muchos otros sistemas distribuidos, 
tiene la ventaja de que es
resistente a fallas, ya que cada nodo guarda toda 
la información del sistema. A diferencia de modelos
cliente-servidor, no hay ningún nodo privilegiado.

