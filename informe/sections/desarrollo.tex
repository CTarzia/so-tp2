
\section{Desarrollo}



%Importante: Desarrolle un análisis del protocolo descrito en este trabajo que responda, al menos, a las siguientes preguntas:

%¿Puede este protocolo producir dos o más blockchains que nunca converjan?

%¿Cómo afecta la demora o la pérdida en la entrega de paquetes al protocolo?

%¿Cómo afecta el aumento o la disminución de la dificultad del Proof-of-Work a los conflictos entre nodos y a la convergencia? Pruebe variando la constante DEFAULT_DIFFICULTY para adquirir una intuición.


Para generar nuestro \textit{Blockchain} vamos a crear un bloque de \'indice $0$ para todos los nodos desde el que van a empezar a generar nuevos bloques. Luego cada nodo comenzar\'a a minar nuevos bloques con \'indices incrementales empezando desde el $1$. Si bien no es algo que suceda en la vida real para este tipo de algoritmos, como queremos que el nuestro termine, tenemos una cantidad m\'axima de bloques que va a tener la cadena de cada bloque. Una vez alcanzada esta cantidad, el nodo deja de producir nuevos bloques y si no le quedan mensajes por responder, termina su ejecuci\'on.

