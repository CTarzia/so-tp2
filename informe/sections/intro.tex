\section{Introducción}

% Se introduce el problema, se muestra un review
% de la literatura y otros aspectos introductorios como posibles
% aplicaciones.


% \subsection{Se espera}
% Describir detalladamente el problema a resolver dando ejemplos
% del mismo y sus soluciones.


En este trabajo práctico buscamos trabajar con sistemas distribuidos, utilizando la interfaz de envío de mensajes MPI. Particularmente implementaremos un \textit{Blockchain}, es decir, una cadena de bloques enlazados que busca evitar modificaciones a través de un proceso de consenso.

\subsection{Blockchain}
El \textit{Blockchain} consiste en un conjunto de bloques enlazados como los de la Figura \ref{fig:blockchain}, en los que cada uno tiene:
\begin{itemize}
	\item \texttt{\'Indice:} El número de bloque.
	\item \texttt{Dueño:} El identificador de quién creó el bloque.
	\item \texttt{Dificultad:} La cantidad de ceros que debe tener el inicio del hash.
	\item \texttt{Fecha de Creación}
	\item \texttt{nonce:} String para resolver el \textit{Proof-of-Work}
	\item \texttt{Hash del bloque anterior}
	\item \texttt{Hash del bloque}
\end{itemize}

\begin{figure}[H]
	\centering
	\includegraphics[width=0.7\linewidth]{img/blockchain}
	\caption{Representación gráfica de un \textit{Blockchain}.}
	\label{fig:blockchain}
\end{figure}


Para conseguir un bloque, cada nodo del sistema distribuido debe \textit{minarlo}, es decir, debe superar una prueba con un costo de cómputo. A esta se la llama \textit{Proof-of-Work (POW)}, y consiste en variar el campo \texttt{nonce} hasta que la representación del hash empiece con al menos la cantidad de ceros que indica \texttt{dificultad}. 

\subsection{Consenso}
Ya que el objetivo del \textit{Blockchain} es que todos los nodos tengan la misma cadena, y cada uno de ellos está calculando bloques para agregar, estos deben comunicarse cada vez que generan o reciben un bloque. Para todos mantener la misma cadena, cada vez que un nodo mina un bloque exitosamente debe realizar un \textit{broadcast} a todos los demás con toda la información del mismo. Si el nodo que recibe el bloque no puede agregarlo a su cadena, 
debe tomar la decisión de si abandonar su cadena y pedir más bloques al nodo emisor o no. 















